\chapter{Outlook and Future Work}
\label{chap:Outlook}

Many things could still be done to the RespVis library that would not
change its core mechanisms, but would improve both the visualization
consumer's and visualization author's experience. One of the most
apparent improvements would be the addition of further Series, Charts,
and Chart Windows to extend the range of realizable visualizations and
enable the creation of things like parallel coordinates, pie charts,
heatmaps, small multiples, and other charts. In addition to supporting
more types of charts, the RespVis library would benefit from
additional tools which could be added to the Toolbars of Chart Windows
to provide easy access to supplementary operations like interval-based
numerical filtering and zooming. The improvements that could be made
to already existing functionality include improving interactions via
the application of Delaunay triangulation
\parencite{Delaunay,DelaunayAlgorithms} to find the closest
interactable element to the cursor position, improving downloaded SVGs
through optimizing and formatting their document contents, and
improving responsive styling via the application of the newly proposed
CSS Container Queries \parencite{CSSContainment3} when they become
available in browsers.

The layouting of SVG elements could be improved by separating
visualizations into different <div> elements that can be natively laid
out by browsers. With such a layouting mechanism, the custom Layouter
could be removed, and the render process imposed by it, which
effectively forces every laid out element to be rendered twice, would
be unnecessary. The elimination of the custom Layouter and its render
process would improve performance and be much less complex to
implement and understand. The downside of this change would be that
visualizations are not directly rendered as pure and complete SVG
documents anymore, but rather as multiple SVG documents representing
separate parts of the visualization. This separation into multiple SVG
documents would not be a problem for displaying visualizations in
browsers. To download such a visualization, its individual parts could
be merged back into a pure and complete SVG document during an
additional download pre-processing step.

Custom visualizations are rather tedious to create with the current
implementation, because visualization authors must manually set up
their structure, propagate data through the component hierarchy, and
handle the Layouter's render process. The creation of custom
visualizations could be simplified by introducing generic Chart
Windows that would enable the definition of visualizations using a
data structure potentially similar to visualization grammars like Vega
\parencite{Vega}.
% KA TODO or Cicero or Vega-Lite ??
The data structure of generic Chart Windows would have to include the
actual abstract data that should be visualized and define the
transformations of this data into scales, Axes, Legends, and Series.
During rendering, the render functions of such generic Chart Windows
could then create custom visualizations to reflect the configuration
stored in these data structures. In addition to simplifying the
creation of custom visualizations, generic Chart Windows would also
enable responsively changing a visualization's encoding, like, for
example, turning Point Charts into Heatmap Charts.


% KA TODO  small multiples?



