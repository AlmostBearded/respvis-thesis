%----------------------------------------------------------------
%
%  File    :  thesis-macros.tex
%
%  Author  :  Keith Andrews, IICM, TU Graz, Austria
%
%  Created :  27 Apr 1994
%
%  Changed :  19 Feb 2004
%
%----------------------------------------------------------------

% common macros and definitions


% \liintro list item intro is a style used when list items have an
% introduction phrase (say in italics) followed by a colon.
\newcommand{\liintro}[1]{\emph{#1}}


% \liheading list item heading
% when list item has an intro phrase in bold
\newcommand{\liheading}[1]{\textbf{#1}}




% short notes in square brackets
\newcommand{\shortnote}[1]
{%
{{\smaller{}[#1]}}
}


\newcommand{\TODO}[1]
{
{\textcolor{red}{[TODO: #1]}}
}



\newcommand{\imgcredit}[1]
{\smaller{}[#1]}




\newcommand{\copyrightACM}
{%
Copyright \copyright\ by the Association for Computing Machinery, Inc.%
}



% \newcommand{\tsup}[1]{\textsuperscript{#1}}



\newcommand{\chapquote}[2]
{%
\begin{quote}
\emph{%
``#1''%
}%
\begin{flushright}
{\scriptsize \sffamily [#2]}%
\end{flushright}
\end{quote}
}





% require the datetime and fmtcount packages
% \usepackage[short]{datetime}   % load datetime *after* babel, requires fmtcount

% for l2h: copy datetime.perl and fmtcount.perl into styles

% TODO: use new datetime2 instead of datetime

\newcommand{\daymonthyear}[3]
{%
\twodigit{#1}\hspace{0.7ex}\nolinebreak[2]\shortmonthname[#2]\hspace{0.7ex}\nolinebreak[2]#3%
}


\newcommand{\monthyear}[2]
{%
\shortmonthname[#1]\hspace{0.7ex}\nolinebreak[2]#2%
}


\newcommand{\yearmonthday}[3]
{%
\twodigit{#3}\hspace{0.7ex}\nolinebreak[2]\shortmonthname[#2]\hspace{0.7ex}\nolinebreak[2]#1%
}


\newcommand{\yearmonth}[2]
{%
\shortmonthname[#2]\hspace{0.7ex}\nolinebreak[2]#1%
}




% based on url package
% define styles for class, file, and variable names
% which break nicely at line breaks

\newcommand{\ttname}{\begingroup \smaller\urlstyle{tt}\Url}
\newcommand{\rmname}{\begingroup \smaller\urlstyle{rm}\Url}
\newcommand{\sfname}{\begingroup \smaller\urlstyle{sf}\Url}

% make the macros robust so they work inside captions, etc

% fname is for file names and directory names
\newrobustcmd{\fname}[1]{\ttname{#1}}

% vname is for variable names, domain names, email addresses
\newrobustcmd{\vname}[1]{\ttname{#1}}




% for class names, define our own url style

\makeatletter  % protect @ names

% \url@letstyle: New URL style to premit break at any letters.
% Based on \url@ttstyle

\def\Url@letdo{% style assignments for tt fonts or T1 encoding
\def\UrlBreaks{\do\a\do\b\do\c\do\d\do\e\do\f\do\g\do\h\do\i\do\j\do\k\do\l%
               \do\m\do\n\do\o\do\p\do\q\do\r\do\s\do\t\do\u\do\v\do\w\do\x%
               \do\y\do\z%
               \do\A\do\B\do\C\do\D\do\E\do\F\do\G\do\H\do\I\do\J\do\K\do\L%
               \do\M\do\N\do\O\do\P\do\Q\do\R\do\S\do\T\do\U\do\V\do\W\do\X%
               \do\Y\do\Z%
}%
\def\UrlBigBreaks{\do\.\do\@\do\\\do\/\do\!\do\_\do\|\do\%\do\;\do\>\do\]%
 \do\)\do\,\do\?\do\'\do\+\do\=\do\#\do\:\do@url@hyp}%
\def\UrlNoBreaks{\do\(\do\[\do\{\do\<}% (unnecessary)
\def\UrlSpecials{\do\ {\ }}%
\def\UrlOrds{\do\*\do\-\do\~}% any ordinary characters that aren't usually
\Urlmuskip = 0mu plus 1mu%
}

\def\url@letstyle{%
\@ifundefined{selectfont}{\def\UrlFont{\sf}}{\def\UrlFont{\sffamily}}\Url@letdo
}

\makeatother  % unprotect @ names

% class names
\newcommand\letname{\begingroup \smaller\urlstyle{let}\Url}

\newrobustcmd{\cname}[1]{\letname{#1}}



% ui element names
% \newrobustcmd{\uiname}[1]{\letname{#1}}
\newrobustcmd{\uiname}[1]{{\smaller\textsf{#1}}}


% gulp task names
\newrobustcmd{\gtask}[1]{{\smaller\lstinline{#1}}}


% html5 element names
\newrobustcmd{\elname}[1]{{\smaller\lstinline{#1}}}

% html5 attribute names
\newrobustcmd{\attrname}[1]{{\smaller\lstinline{#1}}}

% css class names
\newrobustcmd{\cssname}[1]{{\smaller\lstinline{#1}}}

% command line commands
\newrobustcmd{\cmdname}[1]{{\smaller\lstinline{#1}}}


% selector patterns
\newrobustcmd{\pattname}[1]{{\smaller\lstinline{#1}}}







% link to Amazon or
% http://worldcatlibraries.org/wcpa/isbn/[ISBN number]
% http://amazon.com/exec/obidos/ASIN/#1/keithandrewshcic
% http://amazon.com/dp/#1/

\newrobustcmd{\isbn}[1]
{%
{%
\ifpdf
{\smaller ISBN
\href{http://amazon.co.uk/dp/#1/}{#1}}%
\else
{\smaller ISBN #1}%
\fi
}%
}



% ISSN
% http://www.bl.uk/services/bibliographic/issn.html
% 8 digits, should be printed xxxx-xxxx
% e.g. 0020-0190 is Information Processing Letters, Elsevier
%
% Lookup services:
% http://kmittlib.lib.kmutt.ac.th:81/search/i?SEARCH=0020-0190
% http://worldcatlibraries.org/wcpa/issn/0020-0190

\newrobustcmd{\issn}[1]
{%
{%
\ifpdf
{\smaller ISSN
\href{http://worldcatlibraries.org/wcpa/issn/#1}{#1}}%
\else
{\smaller ISSN #1}%
\fi
}%
}



% DOIs  http://doi.org/  e.g.
% doi:10.1038/nature723
% http://doi.org/10.1038/nature723

\newrobustcmd{\doi}[1]
{%
{%
\def\UrlFont{\smaller\rmfamily}
\ifpdf                                   % pdflatex
\href{https://doi.org/#1}{doi:\protect\nolinkurl{#1}}%
\else                                    % latex
doi:\protect\nolinkurl{#1}%
\fi
}%
}





\newrobustcmd{\website}[1]
{%
\ifpdf                                  % pdflatex
\href{http://#1/}{\nolinkurl{#1}}%
\else                                   % latex
\nolinkurl{#1}%
\fi
}




\newcommand{\news}[1]
{%
\ifpdf
\href{news:#1}{\nolinkurl{#1}}
\else
\nolinkurl{#1}%
\fi
}






% Euro symbol

\newcommand{\euro}{\texteuro\,}


% times symbol
\newcommand{\timessym}{\texttimes\,}


% approx symbol

\newcommand{\approxsym}{\ensuremath\approx\,}


% plusminus symbol

\newcommand{\plusminussym}{\textpm\,}


% not equal symbol

\newcommand{\neqsym}{\ensuremath\neq\,}


% rightarrow symbol

\newcommand{\rightarrowsym}{\ensuremath\rightarrow\,\,}



% thumbs up and thumbs down symbols

\newcommand{\uthumb}{\smaller[2]\raisebox{1pt}{\textcolor{DarkGreen}{\faThumbsUp}}}

\newcommand{\dthumb}{\smaller[2]\raisebox{1pt}{\textcolor{DarkRed}{\faThumbsDown}}}



