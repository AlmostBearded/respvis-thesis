\chapter{Introduction}
\label{chap:Introduction}

The web is an indispensable part of modern society, and with the
ever-growing amounts of available data, visualizations which
effectively communicate this data are an essential part of it.
Motivated by the increasing variety of devices used to access the web,
responsive design has been established as one of the core pillars of
designing web content to ensure that it is easily accessible to users
regardless of the characteristics of their devices. Even though most
web content is already designed responsively, charts and
visualizations are often only embedded in a static form, which does
not adapt or only minimally adapts to different device
characteristics. The academic field of responsive visualization is
still in its early stages, but some research from various authors
regarding scalable visualizations
\parencite{BuildingRespDataVisForTheWeb,LearningRespDataVis}, design
patterns \parencite{RespVis,TechniquesForFlexibleRespVisDesign,
  DesignPatternsTradeOffsRespVis}, and tools
\parencite{TechniquesForFlexibleRespVisDesign,Cicero} has already been
done and continues to emerge.

Many different software libraries to create visualizations for the web
exist, but they all have their shortcomings regarding usability,
extensibility, and responsiveness.  One of the most popular libraries
for creating visualizations is D3, which provides a low-level API to
transform HTML and SVG documents based on data.  This document-based
approach is quite powerful as it allows users to create whatever
visualizations they wish for without being hindered by the limitations
of a custom renderer.  However, building visualizations by manually
setting up their entire structure and behavior can be tedious and
requires deep knowledge about D3 and the underlying rendering standard
like SVG.  Other visualization libraries like Vega \parencite{Vega}
focus on a grammar-based approach to rendering visualizations.
Visualization grammars are very expressive and allow users to focus on
a visualization's high-level specification, but they tend to be rather
complex and are not always easy to understand.  Furthermore, since the
actual rendering is abstracted away, visualization authors are limited
to the capabilities offered by a library's high-level API, which can
lead to configurability restrictions.  Yet other types of
visualization libraries are based on template-based configuration,
meaning that visualization authors only need to provide data in a
predefined format and the library then renders predefined
visualizations using this data.  These template-based visualization
libraries are usually easy to use, but same as with grammar-based
libraries, it can be hard to extend visualizations beyond the intended
configuration capabilities.

This thesis introduces RespVis, a new software library for creating
presentational information visualizations for the web with a strong
emphasis on responsiveness.  RespVis is an open-source library
\parencite{RespVisGitHub} that has been designed as an extension of D3
and focuses on rendering visualizations as SVG documents.  Its API has
intentionally been kept as minimal as possible and only allows the
configuration of data that affects a visualization's content or
behavior.  Instead, CSS is used to style and largely position a
visualization's content.  The main contribution of this work lies in a
custom layouter that uses the browser's own layout engine to enable
the positioning of SVG-based visualization components, which would
otherwise be unaffected by CSS layouting.  Allowing visualization
authors to configure the layout of visualization components with CSS
layouting mechanisms like Flexbox and Grid leads to better responsive
capabilities than merely allowing them to style components with CSS.
Additionally, the usage of CSS for styling and positioning allows the
application of other tools frequently used for responsive design like
media queries, and it also means that styles can easily be configured
and overwritten via the CSS cascade.  Since RespVis mainly renders and
configures visualizations using standardized web technologies such as
SVG and CSS, visualization authors can work with technologies all web
developers are already familiar with and do not have to learn a
complex domain-specific language.  Furthermore, this focus on
standardized web technologies also means that visualizations can
easily be extended beyond foreseen use cases, meaning that it is less
likely that visualization authors are limited by restrictions of the
library's API.

The first part of this thesis, Chapters~\ref{chap:WebTechnologies} to
\ref{chap:ResponsiveInformationVisualization}, offers a broad view on
other works in the field into which this work is
embedded. Chapter~\ref{chap:WebTechnologies} introduces the various
web technologies onto which RespVis has been built, the different ways
of embedding graphics into web pages, and the different layout engines
that have been considered for laying out SVG elements.
Chapter~\ref{chap:InfoVis} gives an overview of the field of
information visualization and its history.  Furthermore, some of the
more popular software libraries like D3, grammar-based libraries like
Vega, and template-based libraries like Highcharts used to create
information visualizations for the web are examined and compared
regarding their capabilities to make visualizations responsive.
Chapter~\ref{chap:ResponsiveInformationVisualization} further
specializes on the research around responsive visualizations.
Specifically, the topic of responsive patterns is introduced, and
their application is demonstrated with concrete examples from both
academic and other sources.

The second part of this thesis, Chapters~\ref{chap:RespVis} to
\ref{chap:Outlook}, discusses the technical details of RespVis.
Chapter~\ref{chap:RespVis} introduces the library, its design pillars,
naming conventions, and project setup.  Chapter~\ref{chap:Modules}
describes RespVis' implementation by examining the different packages
and modules into which the library has been split.  This chapter also
discusses the implementation and implications of the custom layouter
that enables laying out the elements of SVG-based visualizations using
CSS layout mechanisms.  Furthermore, Chapter~\ref{chap:Usage}
demonstrates the usage of RespVis' modules to create responsive
visualizations and explains how different responsive patterns can be
applied to them.  Finally, Chapter~\ref{chap:Outlook} gives an outlook
on upcoming topics and potential future work to be done on the
library.

