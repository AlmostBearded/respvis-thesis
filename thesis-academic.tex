%----------------------------------------------------------------
%
%  File    :  thesis-academic.tex
%
%  Author  :  Keith Andrews, IICM, TU Graz, Austria
% 
%  Created :  27 May 93
% 
%  Changed :  19 Feb 2004
% 
%----------------------------------------------------------------


\chapter{Academic Writing}

\label{chap:Academic}


\chapquote{
Imitation is the sincerest flattery.
}
{
Charles Caleb Colton, English writer, 1780--1832.
}



Writing in an academic context is different to other types of
writing. Care must be taken to follow the conventions of
academic writing.



\section{Academic Criteria}

An academic thesis must demonstrate the following components:
\begin{itemize}
\item Motivation. What problem is being addressed and why.

\item Literature survey. A thorough review of related work in the field.

\item Original work. The original work of the author, and what is new
  about it.

\item An extensive bibliography. To demonstrate knowledge of the major
  works in the field, even if they have not all been read in their
  entirety.
\end{itemize}
Probably the most important component is to emphasise and underline
the author's own original contribution, i.e.\ how the work has
contributed to advancing the field.



\section{Academic Integrity}

It is very easy to find helpful material on the web. Resist the
temptation to copy such material verbatim, even with minor changes in
phrasing and word order. It is just as easy for a supervisor or
advisor (or anyone else for that matter) to check the originality of a
piece of text by copying a passage into Google or services such as
\parencite{PlagiarismOrg}.

Work submitted for academic assessment must be original and created by
the stated author(s). Care must be taken to avoid both
\emph{plagiarism} and \emph{breach of copyright}:
\begin{itemize}
\item \liintro{Plagiarism}: Using the work of others \emph{without
  acknowledgement}.

\item \liintro{Breach of copyright}: Using the work of others
  \emph{without permission}.
\end{itemize}





\subsection{Plagiarism}

Plagiarism is a violation of intellectual honesty. This means copying
other people's work or ideas without due acknowledgement, thus giving
the reader the impression that these are original (your own) work and
ideas. The Concise Oxford Dictionary, 8th Edition, defines plagiarism
as:
\begin{displayquote}
\enquote{
\textbf{plagiarise}
\textbf{1} take and use (the thoughts, writings, inventions, etc.\ of
another person) as one's own. \textbf{2} pass off the thoughts etc.\
of (another person) as one's own.
}
\end{displayquote}
Plagiarism is the most serious violation of academic integrity and can
have dire consequences, including suspension and expulsion
\parencite{Reisman2005}.



\subsection{Breach of Copyright}

Copyright law\footnote{Disclaimer: I am not a lawyer. The comments
  here reflect the situation to the best of my knowledge at the time
  of writing, but do not constitute legal advice. Laws sometimes
  change and I make no guarantees.} varies in detail from country to
country, but certain aspects are internationally widely accepted. In
general, the creator of a work, say a piece of writing, a diagram, a
photograph, or a screenshot, automatically has copyright of that
work. Copyright usually expires 70 years after the creator's
death. The copyright holder can grant the right for others to use or
publish their work on an exclusive or non-exclusive basis.

The copyright laws of most countries generally have provisions for
quoting small parts of a work. Austrian copyright law \parencite[§
  42f]{UrhG} allows for reasonable amounts of text to be quoted in
other works. It does not cover ``quoting'' entire images.




\section{Acceptable Use}

Academic work almost always builds upon the work of others, and it is
appropriate, indeed essential, that related and previous work by
others be discussed in an academic thesis. However, this must be done
according to the rules of acceptable use. There are two forms of
acceptable use:
\begin{itemize}
\item \liintro{Paraphrasing (Indirect Quotation)}: Summarising the ideas
  of someone else using original words and with attribution.
\item \liintro{Quoting (Direct Quotation)}: Including an exact
  verbatim copy inside quotation marks and with attribution.
\end{itemize}
Attribution means that the original source is cited.
Regardless of whether permission has been obtained from the copyright
owner or material is being used under the provisions of a specific
country's copyright law: whenever someone else's work is being used,
academic integrity dictates that the original source must be cited!
%
For further information on acceptable and non-acceptable academic
practice see \parencite{FremdeFedern,Wikipedia-Zitat}.




\subsection{Paraphrasing (Indirect Quotation)}

Paraphrasing means closely summarising and restating the ideas of
another person, but in (your own) original words. When writing a
literature survey, the relevant parts of each paper or source are
generally \emph{paraphrased}. One good technique for paraphrasing is:
\begin{enumerate}
\item Read the original source.
\item Put it down away from view.
\item \emph{Without refering to the original}, summarise it in your own words.
\end{enumerate}
When paraphrasing someone else's ideas, the original source must
always be cited!

Since paraphrased text is not enclosed in quotation marks, it is not
always obvious how to indicate the extent of the text which
corresponds to a particular citation. If the paraphrased text only
covers a single paragraph, include the citation either within or at
the end of the first sentence of the paragraph, or at the end of the
paragraph. Otherwise, describe the extent of the citation in words at
the beginning, for example: This section is based on the work of
\textcite{InfoSkyIVS}.





\subsection{Quoting Text (Direct Quotation)}

In some circumstances, it makes sense to directly \emph{quote} small
parts of text (typically a few sentences or paragraphs) from a
relevant source. When quoting directly, the \emph{exact} words,
spelling, and punctuation of the original are copied verbatim and
enclosed in quotation marks.

Most of an academic paper or thesis must be in words written by the
author(s) themselves. However, when an exact phrase or specific
wording from another source is important, then a direct quotation
should be used. In any case, the original source must be cited!

Short pieces of text can be quotes inline using the \vname{textquote}
command. For example, \textcite{DataAnalysisChallenges} define visual
analytics as an: \textquote{iterative process that involves collecting
  information, data preprocessing, knowledge representation,
  interaction, and decision making.}
%
Longer pieces of quoted text should be put into a \vname{displayquote}
environment. For example, as \textcite[page~99]{HarInfoVis}
explains:
\begin{displayquote}
\enquote{
Information in Hyper-G may be structured both hierarchically into
so-called \emph{collections}, and by means of associative hyperlinks.
A special kind of collection called a \emph{cluster} groups logically
related or multilingual versions of documents. Every document and
collection must belong to at least one collection, but may belong to
several. Navigation may be performed down through the collection
hierarchy (the collection \enquote{hierarchy} is, strictly speaking, a
directed acyclic graph), access rights assigned on a
collection-by-collection basis, and the scope of searches restricted
to particular sets of collections. Collections may span multiple
Hyper-G servers, providing a unified view of distributed resources.

Links in Hyper-G are stored in a separate link database and are
bidirectional (directed, but may be followed backwards): both the
incoming \emph{and} outgoing hyperlinks of a document are always
known and available for visualisation. Furthermore, Hyper-G has fully
integrated search facilities including full text search with relevance
scores and some limited support for similarity measures between
documents.

All in all, the richness of the Hyper-G data model provides plenty of
scope upon which to base visualisations: hierarchical structure,
(bidirectional) hyperlinks, and search and retrieval facilities. The
Harmony client for Hyper-G exploits this richness to provide
tightly-coupled two- and three-dimensional visualisation and
navigational facilities help provide location feedback and alleviate
user disorientation.
}
\end{displayquote}





\subsection{Quoting Images}

It is common to want to include photographs, diagrams, or screenshots
taken from the internet or from another work, particularly when
surveying related work. By default, it must be assumed that such
images are covered by copyright and \emph{cannot} simply be used.
Explicit permission \emph{must} be obtained for each image.

Sometimes, permission is granted in advance by the owner in the form
of a licence, such as one of the Creative Commons licences
\parencite{CC-Licences}. Other times, permission can be obtained
directly from the owner by sending a friendly email request. Without
permission, the image \emph{cannot} be used.

Once copyright has expired (in general, 70 years after the death of
the creator), an image passes into the public domain. However, even if
a rare original historical work may technically be in the public
domain, the owner of such a work controls access to it, and has
copyright over any photographs or scans of the work which they create.


For diagrams, an alternative strategy is to redraw and possibly adapt
the diagram in a (vector graphics) drawing editor such as Adobe
Illustrator \parencite{Adobe-Illustrator} or Inkscape
\parencite{Inkscape}. The original source should be cited with wording
like ``Redrawn from Figure N of [\ldots].'' or ``Adapted from Figure N
of [\ldots].''.

For graphs and plots, it is often possible to reconstruct the graphic
from the original data using tools such as gnuplot \parencite{gnuplot}
or R \parencite{R-Project}. The original source should be cited with
wording similar to ``Created from the original data [\ldots] using XY
[\ldots].''.


For screenshots of software, it is sometimes possible to obtain the
original software, install it, and make new screenshots. If possible,
an original, local dataset should be used rather than the default (or
a provided) dataset, so that the resulting screenshots are
demonstrably new and unique.
%
In the case of an online tool (running locally in a web browser), a
local original dataset should be loaded if possible. At a minimum, the
default view should be changed, so the resulting screenshot is new and
unique.
%
In both cases, the source of the software should be cited with wording
similar to ``Screenshot of XY [\ldots] created by the author of this
thesis.''.







\subsection{Attribution and Permission}

In general terms, for material included wholesale from elsewhere, two
pieces of information must be clearly stated:
\begin{enumerate}
\item \liintro{Attribution}: The original source of the material must
  be cited.

\item \liintro{Permission}: The terms under which the material is
  being used must be explained. For example, give the \emph{exact}
  Creative Commons licence \parencite{CC-Licences}, state the
  \emph{exact} legal exemption, or state that permission has kindly
  been given by the named original author.
\end{enumerate}

Attribution and permission should be stated in two places:
\begin{itemize}
\item \liintro{Caption}: At the end of the caption of a figure
  or listing containing the material.

\item \liintro{Credits}: In the Credits section at the
  front of the thesis.
\end{itemize}

All this means, of course, that if a thesis is based upon this
skeleton \parencite{KeithThesis}, then the source and permission
should be stated at the appropriate place (in this case, in the
Credits section).







\section{References}

Modern \LaTeXe installations use BibLaTeX \parencite{BibLaTeX} and
Biber \parencite{Biber} to maintain and process references. Much of
the syntax and many of the conventions were carried over from the
original BibTeX \parencite{BibTeX} format, but BibTeX is now obsolete.

Typically, one or more \vname{.bib} files are prepared, containing one
entry for each original source or reference.
Listing~\ref{list:BibFile} shows four typical entries from a
\vname{.bib} file. The \vname{inproceedings} entry describes a paper
published in conference proceedings, the \vname{article} entry
describes a paper published in a journal, and the \vname{booklet}
entry is being used for internet resources and web sites
(\vname{booklet} has the advantage over \vname{online} that it has a
\vname{howpublished} field.). Every entry type and field type is
documented in the BibLaTeX manual \parencite{BibLaTeX}. The BibLaTeX
Cheat Sheet \parencite{Biblatex-Cheatsheet} provides a convenient
overview.


\begin{samepage}
\lstinputlisting[%
  float=tp,
  aboveskip=\floatsep,
  belowskip=\floatsep,
  xleftmargin=0cm,              % no extra margins for floats
  xrightmargin=0cm,             % no extra margins for floats
%
  language=biblatex,
  basicstyle=\footnotesize\ttfamily,
  frame=shadowbox,
  numbers=left,
  label=list:BibFile,
  caption={[Four Typical Entries from a \vname{.bib} File]%
Four typical entries from a \vname{.bib} file for use
with biblatex and biber.
An \vname{inproceedings} entry describes a paper published
in conference proceedings, an \vname{article} entry describes
a paper published in a journal, and a \vname{booklet} entry
is used for internet resources and web sites.
The \vname{doi} field gives
the DOI (digital object identifier) of the paper.},
]
{listings/some.bib}
\end{samepage}


Of particular note is the \vname{doi} field, which gives the DOI
(digital object identifier) of a paper. DOIs are for academic papers
what ISBNs are for books; a unique handle with which one can easily
find the original. Most publishers are now assigning DOIs to new
conference and journal papers and are working back in time to assign
them to previously published papers. Always give the DOI of a paper
where one is available. If a DOI exists but points to a subscription
site, and the paper is also freely available on the web (say at the
home page of an author), then use the \vname{url} field to give the
free URL as well. Do not redundantly give the same URL in the
\vname{url} field which the DOI itself resolves to.





\subsection{Cleaning Downloaded Bib Entries}

\begin{samepage}
\begin{lstlisting}[%
  float=tp,
  aboveskip=\floatsep,
  belowskip=\floatsep,
  xleftmargin=0cm,              % no extra margins for floats
  xrightmargin=0cm,             % no extra margins for floats
  language=biblatex,
  basicstyle=\footnotesize\ttfamily,
  frame=shadowbox,
  numbers=left,
  label=list:BibACMIEEE,
  caption={[Massaging Bib Entries from ACM and IEEE]%
Bib entries copied from the ACM Digital Library or the
IEEE Computer Society Digital Library contain useful information,
but cannot be used ``as-is''. They must be edited to conform
to biblatex and to these thesis guidelines.
},
]
% From the IEEE Computer Society DL:

@article{10.1109/INFOVIS.2005.7,
author = {Martin Wattenberg},
title = {Baby Names, Visualization, and Social Data Analysis},
journal = {infovis},
volume = {0},
year = {2005},
issn = {1522-404x},
pages = {1},
doi = {http://doi.ieeecomputersociety.org/10.1109/INFOVIS.2005.7},
publisher = {IEEE Computer Society},
address = {Los Alamitos, CA, USA},
}


% From the ACM DL:

@inproceedings{1106568,
 author = {Martin Wattenberg},
 title = {Baby Names, Visualization, and Social Data Analysis},
 booktitle = {INFOVIS '05: Proceedings of the Proceedings of the 2005 IEEE Symposium on Information Visualization},
 year = {2005},
 isbn = {0-7803-9464-x},
 pages = {1},
 doi = {http://dx.doi.org/10.1109/INFOVIS.2005.7},
 publisher = {IEEE Computer Society},
 address = {Washington, DC, USA},
 }


% Clean, edited version for Keith:

@inproceedings{WattenbergNames,
  author       = "Martin Wattenberg",
  title        = "Baby Names, Visualization, and Social Data Analysis",
  booktitle    = "Proc.\ {IEEE} Symposium on Information Visualization
                  (InfoVis 2005)",
  venue        = "Minneapolis, Minnesota, USA",
  organization = "{IEEE} Computer Society",
  isbn         = "078039464X",
  date         = "2005-10",
  pages        = "1--8",
  doi          = "10.1109/INFOVIS.2005.7",
  url          = "http://hint.fm/papers/final-baby-margin-nocomments.pdf",
}

\end{lstlisting}
\end{samepage}


When \vname{.bib} entries are downloaded or copied from the ACM
Digital Library, the IEEE Digital Library, or other onlibne sources,
they should \emph{not} be used as is. They generally need to be
cleaned up first and made consistent with BibLaTeX.
Listing~\ref{list:BibACMIEEE} shows typical BibTeX entries provided by
the ACM Digital Library and the IEEE Computer Society Digital Library.


To bring bib entries into line with biblatex and the examples shown in
Listing~\ref{list:BibFile}, the following should be addressed:
\begin{itemize}
\item The title of the paper should use capitalised main words.

\item Capitalisations in the title which need to be preserved (such as
  the R in VRwave) should be enclosed in curly brackets ({VRwave}).

\item The \vname{title} and \vname{booktitle} should use
  capitalised main words (not all lower case).

\item The \vname{edition} field is usually be a number in inverted
  commas, such as \verb|"2"|, instead of a word such as
  \verb|"Second"|.

\item The name of a conference should be rephrased, with the short
  form of the conference name in parentheses at the end (InfoVis
  2005).

\item Any \vname{year}, \vname{month}, and \vname{day}
  fields should be combined into a \vname{date} field.

\item For a conference paper, the first day of the conference
  should be used as the date of publication.

\item The location of a conference should be in the \vname{venue}
  field, not in the \vname{address} or \vname{location} field. The
  \vname{address} field is for the address of the publisher, but is
  often unnecessary.


\item Any minus signs must be removed from the ISBN number.
  Otherwise, the macro used in this skeleton for handling ISBNs and
  linking to Amazon will break.

\item Any initial \vname{http://doi.acm.org/} or
  \vname{http://doi.ieeecomputersociety.org/} must be removed from
  the DOI. They are \emph{not} part of the DOI.

\item If a free, unofficial version of a paper with a DOI is available
  at the web site of one of the authors, give this in the \vname{url}
  field.

\item Manually shorten any URL as much as possible: try selectively
  removing parameters after a question mark and try removing
  \vname{www} from the domain. Do \emph{not} use a URL shortening
  service like \website{bit.ly}, since there is no guarantee the
  service will be around long term. It is acceptable to use a URL
  shortening service maintained by the original site themselves, such
  as \website{youtu.be} for YouTube URLs.

\end{itemize}







\subsection{What to Reference}

The set of references should be balanced:
\begin{itemize}
\item Do not have largely web sites as references. A few web sites as
  references is fine, most references being web sites is (usually) not
  so good.

\item Do not have too many Wikipedia references. A few Wikipedia
  references is OK; more than a few is not. Wikipedia is a good
  \emph{starting} point for (further) academic research, it is not a
  good ending point for academic research.

\item Have plenty of academic conference and journal papers (with a
  DOI). Sometimes, both an academic paper and a project web site will
  be avilable -- reference both as separate entries.

\item Include some books (with an ISBN) if at all possible. Books
  still count in academic circles.
 
\item If you know or suspect who will be reviewing or marking your
  thesis or paper, make sure to include some of their references. The
  first thing many reviewers do is check to see if they appear in the
  bibliography.

\item No ghost references. Every reference in the bibliography should
  be cited somewhere in the text.

\end{itemize}






\subsection{Citing}

When a citation is included within flowing text:
\begin{itemize}
\item Distinguish between \emph{parenthetical} and \emph{textual}
  citations. Parenthetical citations are used at the end of a
  sentence. Textual citations are used to embed the authors' names in
  the current sentence. For example:

\begin{small}
\hspace{2\parindent}
\renewcommand{\arraystretch}{1.5}
\begin{tabular}{p{0.45\linewidth}p{0.45\linewidth}}
\lstinline|\parencite{InfoSkyIVS}|        & \parencite{InfoSkyIVS}. \\
\lstinline|As \textcite{InfoSkyIVS} say,| & As \textcite{InfoSkyIVS} say,
\end{tabular}
\end{small}



\item If one specific part in a long paper or book is being cited,
  always state the page number or page range in the citation:

\begin{small}
\hspace{2\parindent}
\renewcommand{\arraystretch}{1.5}
\begin{tabular}{p{0.45\linewidth}p{0.45\linewidth}}
\lstinline|\parencite[pages 173--174]{InfoSkyIVS}|        &
  \parencite[pages 173--174]{InfoSkyIVS}. \\
\lstinline|As \textcite[pages 173--174]{InfoSkyIVS} say,| &
  As \textcite[pages 173--174]{InfoSkyIVS} say,
\end{tabular}
\end{small}



\item Multiple sources can be combined into one citation
command:

\begin{small}
\hspace{2\parindent}
\renewcommand{\arraystretch}{1.5}
\begin{tabular}{p{0.45\linewidth}p{0.45\linewidth}}
\lstinline|\parencites{InfoSkyStudies}[pages 173--174]{InfoSkyIVS}| &
  \parencites{InfoSkyStudies}[pages 173--174]{InfoSkyIVS}. \\
\lstinline|As \textcites{InfoSkyStudies}[pages 173--174]{InfoSkyIVS} say,| &
  As \textcites{InfoSkyStudies}[pages 173--174]{InfoSkyIVS} say,
\end{tabular}
\end{small}


\end{itemize}


Here are two examples embedded into some running text. The InfoSky
\parencite{InfoSkyIVS} system combined both hierarchical visualisation
and placement by similarity.
\textcite[Chapter~7]{InteractiveDataVisualisation} categorise
visualisation techniques for multi-variate data according to the
graphical primitive used in the rendering: points, lines, and regions.






% TODO
% https://www.cs.ubc.ca/~tmm/courses/cheat.html



