\chapter{Concluding Remarks}
\label{chap:Concl}

After giving an overview of related web technologies and the research related to the academic fields of information visualizations and responsive visualizations, this thesis introduced RespVis, a new open-source software library to create responsive visualizations for the web.
RespVis has been designed as an extension of the D3 library and renders visualizations as SVG documents styled with CSS.
The most novel contribution of this work is a custom layouter that uses the browser's own layout engine to enable visualization authors to configure the layout of SVG-based visualization components via CSS.
Since rearranging content is one of the main techniques of responsive web design, enabling visualization authors to use CSS layout mechanisms like Flexbox and Grid to reposition visualization components leads to much better responsive capabilities than merely allowing them to change their styles.
Relying on CSS for a large amount of a visualization's configuration also leads to visualization authors benefitting from being able to utilize CSS media queries for responsive styling and from the simplicity of using a tool they are already familiar with.
Furthermore, since RespVis' API is mostly meant for configuring a visualization's content and behavior, it can be much more minimal than it would be if the complete style of a visualization would also be configured via it.
This minimal API and RespVis' reliance on standards like SVG and CSS for rendering and configuring its visualizations result in much less likelihood that visualization authors are limited by API restrictions.
Due to all of these reasons, it is evident that RespVis has the potential to be a very effective library for the creation of responsive visualizations after some more improvements are made to it.
