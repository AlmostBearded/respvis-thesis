%----------------------------------------------------------------
%
%  File    :  thesis-style.tex
%
%  Author  :  Keith Andrews, ISDS, TU Graz, Austria
% 
%  Created :  27 May 1993
% 
%  Changed :  05 Nov 2018
% 
%----------------------------------------------------------------


\chapter{Language and Writing Style}
\label{chap:Style}


\chapquote{
It is an old observation that the best writers sometimes disregard the
rules of rhetoric. When they do so, however, the reader will usually
find in the sentence some compensating merit, attained at the cost of
the violation.
}
{
William Strunk, Jr., The Elements of Style, 1918.
}


A comprehensive guide to writing British English is the New Oxford
Style Manual \parencite{NewOxfordStyleManual-3Ed}. The Economist Style
Guide \parencite{EconomistStyleGuide-12Ed} provides a compact indexed guide
to British English usage. \textcite{Zinsser-OnWritingWell-7Ed} is an
easy to read companion.

A comprehensive guide to writing American English is the Chicago
Manual of Style \parencite{ChicagoManualStyle-17Ed}.  The classic
compact reference for American English writing style and grammar is
\textcite{StrunkWhite-4Ed}. The original text is now available for
free online \parencite{Strunk-1Ed}. Another good free guide is
\textcite{NASAGuide}.

\textcite{Alley-CraftScientificWriting-4Ed} is a classic guide to
scientific writing. Other good ones include
\textcite{Booth-CraftResearch-4Ed} and
\textcite{Booth-CommunicatingScience-2Ed}.
%
\textcite{Zobel-WritingCompSci} and \textcite{BugsInWriting} are guides
specifically aimed at computer science students.
\textcite{Phillips-HowGetPhD} gives practical advice for PhD
students.
%
In 2017, Google made its internal Documentation Style Guide
public \parencite{GoogleStyleGuide}.


Sections~\ref{sec:Clear} and \ref{sec:Gender} of this chapter are
adapted from the ACM CHI'94 conference language and writing style
guidelines.



% TODO
% https://www.cs.ubc.ca/~tmm/writing.html






\section{Paragraphs}

Sentences should be grouped into paragraphs by topic. A new paragraph
introduces a (slight) variation in topic. Paragraphs should consist of
\emph{several} sentences. In general, short paragraphs of only one or
two sentences should be merged topically with neighbouring paragraphs.
%
In \LaTeX, paragraphs are separated by a blank line. Random newlines
({\smaller\verb|\newline|} or {\smaller\verb|\\|}) should \emph{never}
be strewn throughout your text.

% In LaTeX, use a single-line comment to merge two short, but
% topically connected, paragraphs into one paragraph.





\section{Some Basic Rules of English}

There are a few basic rules of English for academic writing, which are
broken regularly by my students, particularly if they are non-native
speakers of English. Here are some classic and often encountered
examples:

\begin{itemize}[itemsep=2ex]

\item \emph{Never} use I, we, or you.

Write in the passive voice (third person).

\begin{tabular}{lp{0.9\linewidth}}
\dthumb & You can do this in two ways.   \\
\uthumb & There are two ways this can be done.  \\
\end{tabular}



\item \emph{Never} use he or she, his or her.

Write in the passive voice (third person).

\begin{tabular}{lp{0.9\linewidth}}
\dthumb & The user speaks his thoughts out loud.   \\
\uthumb & The thoughts of the user are spoken out loud.  \\
\end{tabular}

See Section~\ref{sec:Gender} for many more examples.



\item Stick to a consistent dialect of English. Choose either
  British or American English and keep to it throughout the
  whole of your thesis.



\item Do \emph{not} use slang abbreviations such as ``it's'',
  ``doesn't'', or ``don't''.

Write the words out in full: ``it is'', ``does not'', and ``do not''.

\begin{tabular}{lp{0.9\linewidth}}
\dthumb & It's very simple to\ldots       \\
\uthumb & It is very simple to\ldots       \\
\end{tabular}



\item Do \emph{not} use abbreviations such as ``e.g.'' or
  ``i.e.''. 

Write the words out in full: ``for example'' and ``that is''.

\begin{tabular}{lp{0.9\linewidth}}
\dthumb & \ldots in a tree, e.g. the items\ldots        \\
\uthumb & \ldots in a tree, for example the items\ldots   \\
\end{tabular}



\item Do \emph{not} use slang such as ``a lot of''.

\begin{tabular}{lp{0.9\linewidth}}
\dthumb & There are a lot of features\ldots       \\
\uthumb & There are many features\ldots       \\
\end{tabular}



\item Do \emph{not} use slang such as ``OK'' or ``big''.

\begin{tabular}{lp{0.9\linewidth}}
\dthumb & \ldots are represented by big areas.    \\
\uthumb & \ldots are represented by large areas.  \\
\end{tabular}



\item Do \emph{not} use slang such as ``gets'' or ``got''.

Use ``becomes'' or ``obtains'', or use the passive voice (third
person).

\begin{tabular}{lp{0.9\linewidth}}
\dthumb & The radius gets increased\ldots       \\
\uthumb & The radius is increased\ldots         \\
\end{tabular}

\begin{tabular}{lp{0.9\linewidth}}
\dthumb & The user gets disoriented\ldots       \\
\uthumb & The user becomes disoriented\ldots    \\
\end{tabular}




\item \emph{Never} start a sentence with ``But''.

Use ``However,'' or ``Nevertheless,''. Or consider joining the
sentence to the previous sentence with a comma.

\begin{tabular}{lp{0.9\linewidth}}
\dthumb & But there are numerous possibilities\ldots       \\
\uthumb & However, there are numerous possibilities\ldots  \\
\end{tabular}



\item \emph{Never} start a sentence with ``Because''.

Use ``Since'', ``Owing to'', or ``Due to''. Or turn the two
halves of the sentence around.




\item \emph{Never} start a sentence with ``Also''. Also should
be placed in the middle of the sentence.

\begin{tabular}{lp{0.9\linewidth}}
\dthumb & Also the target users are considered. \\
\uthumb & The target users are also considered. \\
\end{tabular}



\item Do \emph{not} use ``that'' as a connecting word.

Use ``which''.

\begin{tabular}{lp{0.9\linewidth}}
\dthumb & \ldots a good solution that can be computed easily.  \\
\uthumb & \ldots a good solution which can be computed easily.  \\
\end{tabular}




\item Do \emph{not} write single-sentence paragraphs. 

Avoid writing two-sentence paragraphs. A paragraph should contain at
least three, if not more, sentences.


\end{itemize}



% rules on the use of a comma in lists
% http://en.wikipedia.org/wiki/Serial_comma







\section{English Usage}
\label{sec:EnglishUsage}

I see these mistakes time and time again. Please do not
let me read one of them in your work.



\begin{itemize}[itemsep=2ex]


\item ``allows to'' is not English.

\begin{tabular}{lp{0.9\linewidth}}
\dthumb & The prototype allows to arrange components\ldots \\
\uthumb & The prototype supports the arrangement of components\ldots \\[1ex]
\dthumb & The system allows to identify issues\ldots \\
\uthumb & Issues can be identified by the system\ldots \\
\end{tabular}

% they allow to achieve



\item ``enables to'' is not English.

\begin{tabular}{lp{0.9\linewidth}}
\dthumb & it enables to recognise meanings\ldots \\
\uthumb & it enables the recognition of meanings\ldots \\
\end{tabular}



\item ``per default'' is not English.

Use ``by default''.

\begin{tabular}{lp{0.9\linewidth}}
\dthumb & Per default, the cursor is red. \\
\uthumb & By default, the cursor is red. \\
\end{tabular}




\item ``As opposed to'' is not English.

Use ``In contrast to''.

\begin{tabular}{lp{0.9\linewidth}}
\dthumb & As opposed to C, Java is object-oriented. \\
\uthumb & In contrast to C, Java is object-oriented. \\
\end{tabular}







\item ``actual'' \neqsym ``current'' 

If you mean ``aktuell'' in German, you probably mean
``current'' in English.

\begin{tabular}{lp{0.9\linewidth}}
\dthumb & The actual selection is cancelled. \\
\uthumb & The current selection is cancelled. \\
\end{tabular}



\item ``sensible'' \neqsym ``sensitive'' 

If you mean ``sensibel'' in German, you probably mean
``sensitive'' in English.

\begin{tabular}{lp{0.9\linewidth}}
\dthumb & Store sensible data securely. \\
\uthumb & Store sensitive data securely. \\
\end{tabular}




\item ``according'' \neqsym ``corresponding'' 

\begin{tabular}{lp{0.9\linewidth}}
\dthumb & For each browser, an according package is created. \\
\uthumb & For each browser, a corresponding package is created. \\
\end{tabular}




\item ``adopt'' \neqsym ``adapt'' 

To ``adopt something'' means ``etwas übernehmen'' in German.
To ``adapt something'' means ``etwas anpassen'' in German.

\begin{tabular}{lp{0.9\linewidth}}
\dthumb & This convention was adapted to show\ldots \\
\uthumb & This convention was adopted to show\ldots \\[1ex]
\dthumb & The diagram was adopted by the author. \\
\uthumb & The diagram was adapted by the author. \\
\end{tabular}



% countable and uncountable nouns
% https://english.stackexchange.com/questions/9439/amount-vs-number-vs-quantity
% https://dictionary.cambridge.org/grammar/british-grammar/amount-of-number-of-or-quantity-of

\item ``amount'' versus ``number'' 

Use ``number'' for countable things.
Use ``amount'' for uncountable things.

\begin{tabular}{lp{0.9\linewidth}}
\dthumb & The amount of students\ldots \\
\uthumb & The number of students\ldots \\[1ex]
\dthumb & The number of time\ldots \\
\uthumb & The amount of time\ldots \\
\end{tabular}



\item ``many'' versus ``much'' 

Use ``many'' for countable things.
Use ``much'' for uncountable things.

\begin{tabular}{lp{0.9\linewidth}}
\dthumb & Much students failed\ldots \\
\uthumb & Many students failed\ldots \\[1ex]
\dthumb & Many time was spent\ldots \\
\uthumb & Much time was spent\ldots \\
\end{tabular}



\item ``fewer'' versus ``less'' 

Use ``fewer'' for countable things.
Use ``less'' for uncountable things.

\begin{tabular}{lp{0.9\linewidth}}
\dthumb & Less participants succeeded\ldots \\
\uthumb & Fewer participants succeeded\ldots \\[1ex]
\dthumb & Fewer sand was blown away\ldots \\
\uthumb & Less sand was blown away\ldots \\
\end{tabular}





\item ``\emph{anything}-dimensional'' is spelt with a hyphen.

For example: two-dimensional, three-dimensional.


\item ``\emph{anything}-based'' is spelt with a hyphen.

For example: tree-based, location-based.


\item ``\emph{anything}-oriented'' is spelt with a hyphen.

For example: object-oriented, display-oriented.


\item ``\emph{anything}-side'' is spelt with a hyphen.

For example: client-side, server-side.


\item ``\emph{anything}-friendly'' is spelt with a hyphen.

For example: user-friendly, customer-friendly.


\item ``\emph{anything}-to-use'' is spelt with hyphens.

For example: hard-to-use, easy-to-use.


\item ``\emph{anything}-level'' is spelt with a hyphen.

For example: low-level, high-level.



\item ``realtime'' is spelt with a hyphen if used as
  an adjective, or as two separate words if used as a noun.

\begin{tabular}{lp{0.9\linewidth}}
\dthumb & \ldots display the object in realtime.  \\
\uthumb & \ldots display the object in real time. \\
\dthumb & \ldots using realtime shadow casting.   \\
\uthumb & \ldots using real-time shadow casting.  \\
\end{tabular}


\end{itemize}












\section{Clear Writing}
\label{sec:Clear}

An academic thesis written in English should use simple and clear
language appropriate for an international audience. In particular:

\begin{itemize}[itemsep=2ex]

\item Write simple, straightforward sentences. Do not use long,
  convoluted sentences with many nested clauses, purely for the whim
  of it, because, as is sometimes the case, it may seem like a good
  idea at the time, even though it is not really.


\item Use common and basic vocabulary. For example:
  \begin{itemize}
  \item ``unusual'' instead of ``arcane''
  \item ``specialised'' instead of ``erudite''.
  \item ``guideline'' instead of ``rule of thumb''.
  \end{itemize}



\item A technical term should be defined once at first usage.  It
  should be placed in italics where it is defined, and in normal
  script whenever used thereafter:

\begin{tabular}{lp{0.9\linewidth}}
\uthumb & A \emph{graph} is a set of vertices and edges.
        A \emph{vertex} (or node) is an individual item. \newline
        An \emph{edge} (or link) is a connection between two vertices.
\end{tabular}

  Any equivalent variant terms should be listed with the
  definition. The preferred term should then be used consistently
  throughout the text, rather than any of the variant terms.
  Otherwise, readers are left wondering whether the variant term
  refers to the same thing or is something different.



\item For generic English text, rather than repeating the same word or
  phrase too often, look in a thesaurus (see
  Section~\ref{sec:thesaurus}) for an alternative word with the same
  meaning.


\item Explain any acronyms the first time they are used, by writing
  out the full phrase followed by the acronym in parentheses.

\begin{tabular}{lp{0.9\linewidth}}
\dthumb & When using SVG, the figure scales freely. \\
\uthumb & When using Scalable Vector Graphics (SVG), the figure scales freely.
\end{tabular}


\item Avoid local references. International readers will probably not
  recognise the names of the provincial capitals of Austria, for
  example. If local context is necessary for understanding, then
  describe it fully.


\item Avoid ``insider'' jargon. Do not assume knowledge of a
  particular context. For example, do not assume the reader is
  familiar with a particular operating system or application.


\item Express culturally localised things such as times, dates,
  currencies, and numbers in an unambiguous form. For example, 9/11 is
  the \nth{9} of November in much of the world. In English, a period
  ``.''  is used as the decimal point character and a comma ``,'' is
  used as the thousands separator (in German, it is the other way
  round).


\item Do not use ``word plays'' or puns. Phrases such as ``red
  herring'', ``taking the mickey'', and ``like watching paint dry''
  require cultural knowledge of English to understand.


\item Be careful with humour. Irony and sarcasm are sometimes hard to
  detect for non-native speakers.

\end{itemize}


Part of writing usable documents is understanding and then addressing
the characteristics of the intended audience.





\section{Avoiding Gender Bias}
\label{sec:Gender}

Two issues should be considered with regard to avoiding gender bias:
avoiding characterisations or stereotypes about men or women,
and avoiding biases inherent in the English language.
Here are some suggestions for handling the second issue:

\begin{itemize}[itemsep=2ex]

\item Refer to people generically using a gender-neutral term:

\begin{tabular}{ll}
   man                 &   the human race        \\
   mankind             &   humankind, people     \\
   manpower            &   workforce, personnel  \\
   man on the street   &   average person        \\
\end{tabular}



\item Use gender-neutral terms for job titles or roles, where
  possible:

\begin{tabular}{ll}
  chairman     &  chairperson \\
  spokesman    &  spokesperson, representative \\
  policeman    &  police officer \\
  stewardess   &  flight attendant \\
\end{tabular}



\item When refering to the holder of a specific position and their
  gender is known, use the correct gender pronoun. For example,
  assuming the chairperson is known to be a man:

\begin{tabular}{lp{0.9\linewidth}}
\dthumb & The chairperson announced her resignation. \\
\uthumb & The chairperson announced his resignation. \\
\end{tabular}



\item Avoid using a gender pronoun by repeating the job title or role
  if possible:

\begin{tabular}{lp{0.9\linewidth}}
\dthumb&
Interview the user first and then ask him to fill out a questionnaire. \\
%
\uthumb &
Interview the user first and then ask the user to fill out a questionnaire. \\
\end{tabular}



\item Avoid using his or her by using the plural form:

\begin{tabular}{lp{0.9\linewidth}}
\dthumb & Each student should bring his text to class. \\
\uthumb & All students should bring their texts to class. \\
\end{tabular}


\item Replace his or her with the article (the):

\begin{tabular}{lp{0.9\linewidth}}
\dthumb & Every student must hand his report in on Friday. \\
\uthumb & Every student must hand the report in on Friday. \\
\end{tabular}



\item Avoid using his or her by rewriting in the passive voice:

\begin{tabular}{lp{0.9\linewidth}}
\dthumb & Each department head should do his own projections. \\
\uthumb & Projections should be done by each department head. \\
\end{tabular}


\item Avoid awkward formulations such as ``s/he,'' ``he/she,'' or
  ``his/her.'' As a last resort, use the less awkward ``he or she,''
  or ``his or hers.''

\end{itemize}






\section{When to Use Capitalisation}

\emph{Capitalisation} means using a capital (upper case) initial
letter for a word. \emph{Lowercasing} means writing the entire word in
lower case. In many common writing styles, headings and titles are
partially capitalised: the first and the principal (main) words are
capitalised and other words are lowercased.

Proper names, such as the names of people, towns, and countries, are
always capitalised (Keith Andrews, the United Kingdom). The first word
in a heading or title is always capitalised.




\subsection{Titles and Headings}

Capitalise all principal words: nouns, pronouns, adjectives, verbs,
and adverbs, and the first word. Lowercase all articles, coordinating
conjunctions (``for'', ``and'', ``nor'', ``but'', ``or'', ``yet'',
``so''), and prepositions.

For example:
\begin{itemize}[itemsep=2ex]

\item Here, ``it'' is a pronoun, which should always be capitalised.

\begin{tabular}{lp{0.9\linewidth}}
\dthumb & Saying it Directly \\
\uthumb & Saying It Directly \\
\end{tabular}



\item Here, ``is'' is a verb, which should always be capitalised.

\begin{tabular}{lp{0.9\linewidth}}
\dthumb & When is Enough Enough? \\
\uthumb & When Is Enough Enough? \\
\end{tabular}



\item Here, ``in'' is being used as a preposition and should be 
lowercased.

\begin{tabular}{lp{0.9\linewidth}}
\dthumb & The Elephant In the Room. \\
\uthumb & The Elephant in the Room. \\
\end{tabular}


\item Here, ``in'' is being used as an adverb and should be 
capitalised.

\begin{tabular}{lp{0.9\linewidth}}
\dthumb & Handing in Your Work. \\
\uthumb & Handing In Your Work. \\
\end{tabular}


\end{itemize}

See \textcite{WB-Capitalisation} for some slightly different rules and
some more examples.


% Capitalization in Titles
% http://www.writersblock.ca/tips/monthtip/tipmar98.htm




\subsection{Captions}

The short version (the optional parameter in square brackets) of a
caption for a figure, table, or listing appears in the List of
Figures, List of Tables, or List of Listings. The short caption is
used like a heading and should be capitalised like a heading. The long
version of a caption for a figure, table, or listing should be written
as full sentences: only the first word of each sentence and any proper
names are capitalised and (each sentence in) the caption ends with a
full stop.



\subsection{Chapters, Sections, Figures, and Tables}

A specific, named or numbered entity, such as a particular chapter,
appendix, section, figure, table, or listing is considered to be a
proper name and thus \emph{should be capitalised}. For example,
Chapter~\ref{chap:SelectedDetails}, Appendix~\ref{app:UserGuide},
Section~\ref{sec:Gender}, Figure~\ref{fig:TeXnicCenter},
Table~\ref{tab:WinIconicLang}, or Listing~\ref{list:BibFile}. However,
if an entity is not specifically named or numbered, then it should
\emph{not} be capitalised. For example, when refering to the first
chapter or the next section, without giving a name or number.


% Capitalization Rules
% http://www.libraryonline.com/default.asp?pID=48

% Chicago Manual of Style
% http://www.chicagomanualofstyle.org/home.html

% Associated Press Stylebook
% http://www.apstylebook.com/







\section{Use a Spelling Checker}

In these days of high technology, spelling mistakes and typos are
inexcusable. It is \emph{very} irritating for your supervisor to have
to read through and correct spelling mistake after spelling mistake
which could have been caught by an automated spelling checker.
Believe me, irritating your supervisor is not a good idea.

So, use a spelling checker \emph{before} you hand in \emph{any}
version, whether it is a draft or a final version.
Since this is apparently often forgotten, and sometimes even wilfully
ignored, let me make it absolutely clear:
\begin{quote}
\begin{em}
Use a spelling checker, please. \\
Use a spelling checker! \\
Use a spelling checker, you moron. \\
\end{em}
\end{quote}





\section{Use a Dictionary}
\label{sec:dictionary}

If you are not quite sure of the meaning of a word, then use a
dictionary. \website{dictionary.com} \parencite{DictionaryCom} is a
free English dictionary, BEOLINGUS \parencite{DictChemnitz} and Leo
\parencite{DictLeoOrg} are two very good English-German dictionaries.




\section{Use a Thesaurus}
\label{sec:thesaurus}

If a word has been used several times already, and using another
equivalent word might improve the readability of the text, then
consult a thesaurus. \website{thesaurus.com} \parencite{ThesaurusCom}
and Collins English Thesaurus \parencite{CollinsThesaurus} are free
English thesauri.


